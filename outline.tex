\newcommand{\thecourse}{Bachelor Project}
\newcommand{\theassignment}{Outline}

\documentclass[11pt, oneside,a4paper]{article}
\usepackage[utf8]{inputenc}
\usepackage[english]{babel}
\usepackage{graphicx}          %include graphics
\usepackage{float}             %better handling of tables, etc.


%mathematics
\usepackage{amssymb,amsmath, amsfonts, amsthm}
\usepackage{mathtools}
\usepackage{extarrows}
\usepackage[intlimits]{esint}

%setup for amsthm
\newtheorem{theorem}{Theorem}[section]
\newtheorem{lemma}[theorem]{Lemma}
\newtheorem{definition}[theorem]{Definition}
\newtheorem{corollary}[theorem]{Corollary}

\theoremstyle{definition}
\newtheorem{example}[theorem]{Example}
\newtheorem{xca}[theorem]{Exercise}
\theoremstyle{remark}

\newtheorem{remark}[theorem]{Remark}


%minor improvements to typesetting
\usepackage[activate={true,nocompatibility},final,tracking=true,kerning=true,spacing=true,factor=1100,stretch=10,shrink=10]{microtype}
\microtypecontext{spacing=nonfrench}
\microtypesetup{protrusion=true}

%adjust margins and headers
\usepackage{geometry,fancyhdr}
\geometry{a4paper,headheight=20pt, footskip=20pt, textheight=684pt, marginparwidth=10pt, textwidth=476pt}
\pagestyle{fancy}
\fancyhf{}
\lhead{\thecourse: \theassignment}
\rhead{Levi Moes S4145135}

\title{\thecourse \theassignment}
\author{Levi Moes}

%commands
\DeclareMathOperator{\spa}{span}
\DeclareMathOperator{\re}{Re}
\DeclareMathOperator{\im}{Im}
\DeclareMathOperator{\Arg}{Arg}
\DeclareMathOperator{\id}{id}
\DeclareMathOperator{\lcm}{lcm}

\begin{document}
\section{Outline}
\label{sec:outline}
Silverman proves
a version of mordell's theorem for curves having a point of order
two.
The goal of this project will be the following.
\begin{theorem} \label{thm:goal}
  Let
  \[ E: y^2 = x^3 + ax^2 + bx + c \]
  be an elliptic curve containing a point of order 3,
  then $E(\mathbb{Q})$ is finitely generated.
\end{theorem}
Potentially we even get the following for free.
\begin{corollary} \label{cor:goalExtra}
  Let
  \[ E: y^2 = x^3 + ax^2 + bx + c \]
  be an elliptic curve containing a point of order 3,
  then $E(\mathbb{Q}(\sqrt{-3}))$ is finitely generated.
\end{corollary}
Our strategy will be to broadly follow Silverman's proof,
namely we will use a specific case of the Descent Theorem
from namely:
\begin{theorem} \label{thm:descentDraft}
  Let $A$ be an abelian group. Suppose
  that there exists a function $h: A \to \mathbb{R}$
  with the following properties.
  \begin{enumerate}
    \item Let $Q \in A$. There exists $C_1(A, Q)$, such that
      for all $P, Q \in A$
      \[ h(P + Q) \leq 2h(P) + C_1(A, Q). \]
    \item There is $C_2(A)$, such that
      for all $P \in A$
      \[ h(3P) \geq 9 h(P) - C_2(A). \]
    \item For every $C_3 \in \mathbb{R}$ the set
      \[ \left\{ P \in A : h(P) \leq C_3 \right\}  \]
      is finite.
  \end{enumerate}
  If furthermore $A/3A$ is finite
  then $A$ is finitely generated.
\end{theorem}
We shall prove this theorem and then prove all its
hypotheses in order of increasing difficulty.
Firstly, we shall define such a function $H: \mathbb{Q} \to \mathbb{R}$.
And then generalise it to elliptic curves.
\begin{definition} \label{def:heightFunctionRationals}
  Let $x = p/q \in \mathbb{Q}$ with $\gcd(p,q) = 1$.
  Then we define the height of $x$ as $H(x) = \max \left\{ |p|, |q| \right\}$.
\end{definition}
And now for Elliptic Curves. We shall understand $E$ to
be an Elliptic Curve.
\begin{definition} \label{def:heightFunctionEllipticCurves}
  Define the height of a rational point $P = (x,y)$ on $E$
  as $H(P) := H(x)$.
\end{definition}
\noindent Now the lemmas.
\begin{lemma} \label{lem:property1}
  For all constants $m \in \mathbb{R}$ we have that the set
  \[ \left\{ P \in E(\mathbb{Q}) : H(P) \leq m \right\} \]
  is finite.
\end{lemma}
\noindent This is a result which holds more generally and its
proof is straightforward.
\begin{lemma} \label{lem:property2}
  Let $P_0 \in E(\mathbb{Q})$, then there exists $b \in \mathbb{R}$ such that
  for every $P \in E(\mathbb{Q})$
  \[ h(P + P_0) \leq 2h(P) + b. \]
\end{lemma}
\noindent Which also holds more generally. The next result
significantly deviates from Silverman's.
\begin{lemma} \label{lem:property3}
  There is $k \in \mathbb{R}$ such that for every point $P \in E(\mathbb{Q})$
  we have
  \[ H(3P) \geq 9H(P) - k. \]
\end{lemma}
\noindent Which is harder to prove. And lastly:
\begin{lemma} \label{lem:property4}
  Suppose $E(\mathbb{Q})$ has a point of order 3.
  Then the subgroup $3E(\mathbb{Q})$ has finite index.
\end{lemma}
\end{document}

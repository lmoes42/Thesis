Throughout this section it shall be well understood that
we are talking about an Elliptic Curve $C: y^2 = f(x)$
over $\mathbb{Q}$ such that $C(\mathbb{Q})$ has a point of order 3.
We shall moreover be using shorthand
\begin{equation*}
  \Gamma := C(\mathbb{Q})
\end{equation*}
This section shall be dedicated to proving the final part of
theorem \ref{thm:3descent}, that is
\begin{theorem} \label{thm:finiteIndex}
  The index $[\Gamma: 3\Gamma]$ is finite.
\end{theorem}
The sketch of our proof will be to define two classes
of curves $E: y^2 = g(\xi)$ and $\bar{E}: y^2 = h(x)$
and define two isogenies $\phi: E(\mathbb{Q}) \to \bar{E}(\mathbb{Q})$
along with $\bar{\phi}: \bar{E}(\mathbb{Q}) \to E(\mathbb{Q})$
such that $\phi \circ \varphi = P \mapsto 3P$.
This gives us an exact sequence
\begin{equation*}
\begin{tikzcd}
  0 \ar[r] & \bar{\phi}\bar{E}/3E \ar[r] & E/3E \ar[r] & E/\bar{\phi}\bar{E} \ar[r] & 0
\end{tikzcd}
\end{equation*}
We find an isomorphism
$f: \bar{\phi}\bar{E}/3E \to \mathbb{Q}(\sqrt{-3})^\times/(\mathbb{Q}(\sqrt{-3})^\times)^3 =: \mathbb{Q}(\sqrt{-3})_3 $
and a homomorphism $g: E(\mathbb{Q}) \to \mathbb{Q}^\times/(\mathbb{Q}^\times)^3 =: \mathbb{Q}_{3}$ with kernel
$\bar{\phi}\bar{E}(\mathbb{Q})$ and show their images are finite.
Which means $E/3E$ must be finite as well.
Lastly we show that it is sufficient
to show any such $3E(\mathbb{Q})$ has
finite index to show theorem \ref{thm:finiteIndex}, where we recall
from theorem \ref{thm:isogenousOrder3} that there are only two
possible forms our equation can have.

\subsection{The Rationals Modulo the 3rd Powers}%
\label{sub:the_rationals_modulo_the_3rd_powers}
A group which we will need to discuss is $\mathbb{Q}_3 := \mathbb{Q}^{\times}/\mathbb{Q}^{\times 3}$.
We can write a typical element of $x \in \mathbb{Q}^{\times}$
as
\[ x = \prod_{i=1}^{\infty} p_i^{d_i} \]
where $p_i$ is the $i$the prime, $d_i \in \mathbb{Z}$ and only finitely
many $d_i$ are nonzero.
Setting $e_i$ as a basis for the infinite product of $\mathbb{Z}/3\mathbb{Z}$
with itself.
Then the following sequence is exact
\begin{equation*}
\begin{tikzcd}
  & & p_i^{d_i} \ar[r, mapsto, "f"] & e_i d_i \\
  0 \ar[r] & \mathbb{Q}^{\times 3} \ar[r, "\iota"] &\mathbb{Q}^{\times} \ar[r, "f"] & \bigoplus_{\mathbb{N}} (\mathbb{Z}/3\mathbb{Z}) \ar[r] & 0.
\end{tikzcd}
\end{equation*}
So from the first isomorphism theorem it follows
\begin{align*}
  \mathbb{Q}_{3} = \mathbb{Q}^{\times} / \mathbb{Q}^{\times 3} \simeq \bigoplus_{\mathbb{N}} \mathbb{Z}/3\mathbb{Z}.
\end{align*}
So without loss of generality, we can assume $x$ is an integer times a
coset, otherwise we just add $3$ to the multiplicity of
$p^{-d}$ until we have a positive multiplicity. Moreover,
since $-1$ is a cube we can assume $x$ can be represented as a positive
integer.

\subsection{Finite Image}%
\label{sub:finite_image}
We define an elliptic curve
\begin{align*}
  E: y^2 = x^3 + a^2(x - b)^2
\end{align*}
Define a map
\begin{equation*}
\begin{tikzcd}
  E(\mathbb{Q}) \ar[r, mapsto, "\alpha"] & \mathbb{Q}_3 \\
  (x, y) \ar[r, "\alpha"] & (y + a(x - b)) \mathbb{Q}^{\times 3}.
\end{tikzcd}
\end{equation*}
Note that $\ker \alpha = E(\mathbb{Q})[3]$.
From the same computation as in example \ref{ex:explicitComputation}
we can write a rational point as $(m/e^2, n/e^3)$ and
similarly to \cite[Section 4.1]{moniqueThesis}
\begin{align*}
  n^2 &= m^3 + a^2 m^2 e^2 - 2a^2 b e^{4} + a^2 b^2 e^{6} \\
  \therefore m^3 &= (n + ame - abe^3)(n - ame + abe^3).
\end{align*}
so our image under $\alpha$ is given as
\begin{align*}
  \alpha(m/e^2, n/e^2) = (n + ame - abe^3)\mathbb{Q}^{\times 3}
\end{align*}
Note that if $n + ame - abe^3$ and $n - ame + abe^3$
are coprime then this point must be a perfect cube and it is
contained in $\ker \alpha$.

Now assume these images do have prime factors in common.
write this as $n + ame - abe^3 = dp$ where $p$ is coprime to $n - ame + abe^3$
and $d = \gcd(n + ame - abe^3, n - ame + abe^3)$.
Now we can follow \cite[Section 4.1]{moniqueThesis}.
\begin{theorem} \label{thm:primeFactors}
  There is a finite set of primes depending only on $a$ and $b$
  such that for any point on the curve the constant $d$ has prime factors
  only from this set.
\end{theorem}
\begin{proof}
  By a standard application of the euclidean algorithm we find
  \begin{align*}
    d &= \gcd(n + ame - abe^3, n - (ame - abe^3)) \\
      &= \gcd(n + ame - abe^3, -2 ame + 2abe^3) \\
      &= \gcd(n + ae(m - be^2), -2ae(m - be^2))
  \end{align*}
  recall that $\gcd(n, e) = 1$. Since we only wish to
  show finitely many prime factors exist we can ignore $-2a$ and simply add
  these factors to our finite set, so we take
  \begin{align*}
    d' := \gcd(n + ae(m - be^2), m - be^2)
  \end{align*}
  and show it has a finite number of
  prime factors
\end{proof}

The remainder of this project will be about finding a height function
suitable for $E(\mathbb{Q})$, and then proving each of the properties
in theorem \ref{thm:3descent}.

We will first discuss a few examples so as to motivate an intuition behind
a height function.
\begin{example}
  In the case that $A = \mathbb{Q}$ we can define a function
  $h_{\mathbb{Q}}: \mathbb{Q} \to \mathbb{R}$
  \begin{equation} \label{eq:rationalHeight}
    h_{\mathbb{Q}}: p/q \mapsto \max \left\{ |p|, |q| \right\}
  \end{equation}
  and while it is known $\mathbb{Q}$ is not finitely generated
  as a group, we can still prove one of the properties from the
  3-descent theorem holds.
  Namely, we know that for fixed $m \in \mathbb{R}$ there
  are only finitely many rational numbers with height bounded
  by $m$. If $h(p/q) \leq m$ then both $p, q \leq m$,
  for which there are only finitely many possibilities.
\end{example}
\begin{definition} \label{def:ellipticHeight}
  (heights on Elliptic Curves)
  If $E: y^2 = f(x)$ is an Elliptic Curve over $\mathbb{Q}$, then
  we define the height of a point $P = (x,y)$ as
  $h(P) = \ln h_{\mathbb{Q}}(x)$, where $h_{\mathbb{Q}}$ is
  as in equation \ref{eq:rationalHeight}.
\end{definition}
The remainder of this section shall be dedicated to proving
this notion of height satisfies theorem \ref{thm:3descent}.

\subsection{Bound on Height}%
\label{sub:bound_on_height}
Here we prove the first property of the 3-descent theorem, namely
\begin{lemma} \label{lem:boundHeight}
  Let $E: y^2 = f(x)$ be an Elliptic Curve.
  Then if $P \in E(\mathbb{Q})$ then for
  every $Q \in E(\mathbb{Q})$ there
  is an integer $C_1$ such that
  \begin{equation} \label{eq:boundOnHeightOfCurve}
    h(P + Q) \leq 2 h(P) + C_1.
  \end{equation}
\end{lemma}
This is found in \cite[section 3.2]{silvermanRationalPoints}.
\begin{proof}
  If $Q = \mathcal{O}$ then this
  is trivial.
  Suppose $Q \neq \mathcal{O}$,
  we prove that for all $P$ except
  for $ P \in \left\{ -Q, Q, \mathcal{O} \right\}$ there is
  $\tilde{C}_1$ such that \ref{eq:boundOnHeightOfCurve} holds.
  Then set $C_1 = \max\{h(-Q), h(Q), h(\mathcal{O}), \tilde{C}_1\}$.
  This allows us to assume the $x$ coordinates of the points
  are different.

  So write $P = (x,y)$ and $Q = (x_0, y_0)$.
  So set $P + Q = (\xi, \eta)$.
  From how we defined the group law
  on elliptic curve \ref{def:groupLaw} we find
  \begin{align*}
    \xi = \frac{(y - y_0)^2 - (x - x_0)^2(x + x_0 + a)}{(x - x_0)^2}
  \end{align*}
  where this is the same $a$ as in the definition of an elliptic curve $y^2 = x^3 + ax^2 + \cdots$.
  Using the relation of the curve we find
  there are rational numbers $A, \dots, G$ such that
  \begin{align*}
    \xi = \frac{Ay + Bx^2 + Cx + D}{Ex^2 + Fx + G}
  \end{align*}
  Without loss of generality, these are all integers, else
  we just multiply with their least common multiple.
  So note that when we fix $P$ there is no dependence on the
  coordinates of $Q$ anymore. So this shall serve for
  our constant $C_1$.

  Using the substitution $x = m/e^2, y = n/e^3$
  we simplify
  \begin{align*}
    \xi = \frac{Ane + Bm^2 + CM^2 + De^4}{Em^2 + Fme^2 + Ge^4}
  \end{align*}
  so indeed
  \begin{align*}
    \exp h_{\mathbb{Q}}(\xi) \leq \max \left\{ |Ane + Bm^2 + Cme^2 + De^4|, |Em^2 + Fme^2 + Ge^4| \right\}
  \end{align*}
  Applying the triangle inequality tells us
  \begin{align*}
    \exp h(P + Q) = H(\xi) \leq \max \left\{ |AK| + |B| + |C| + |D|, |E| + |F| + |G| \right\} \exp h(Q)^2
  \end{align*}
  taking the log of both sides yields the desired result.
\end{proof}

\subsection{Height of 3P}%
\label{sub:height_of_3p}
Now that we have proven the first property, we arrive at the second.
\begin{lemma} \label{lem:height3p}
  There is a constant $C_1$ such that
  \begin{align*}
    h(3P) \geq 9h(P) - C_2
  \end{align*}
\end{lemma}
It can be shown \cite[Appendix B]{moniqueThesis} that
this is just a specific case of
a lemma in Silverman's Book \cite[Lemma 3.6]{silvermanRationalPoints}.
Proving either this lemma or the equivalence is outside the scope of this thesis.

\subsection{Points of Bounded Height}%
\label{sub:points_of_bounded_height}
Now for the 3rd property.
\begin{lemma} \label{lem:pointsOfBoundedHeight}
  For every constant $C_3$ the set
  \begin{align*}
    \left\{ Q \in A : h(Q) \leq C_3 \right\}
  \end{align*}
  is finite.
\end{lemma}
Note that this exact same reasoning also works over $\mathbb{Q}$.
\begin{proof}
  Recall
  \begin{align*}
    h(p/q, y) = \ln \max \left\{ |p|, |q| \right\}
  \end{align*}
  so for any $C_1$ there are only
  finitely many options for $p$ and $q$.
\end{proof}
The final property is significantly harder to prove and we will dedicate
an entire chapter to it.

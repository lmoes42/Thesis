\begin{theorem}
  For any Elliptic Curve $E: y^2 = f(x)$ the group $E(\mathbb{Q})$ is finitely
  generated.
\end{theorem}
Which requires results from cohomology which are beyond the
scope of this thesis \cite[Section VIII]{silvermanArithmetic} .
We shall prove the weaker version
\begin{theorem} \label{thm:mordell}
  Let $E: y^2 = f(x)$ be an Elliptic Curve. If
  $E(\mathbb{Q})$ contains a point of order 3, then
  $E(\mathbb{Q})$ is finitely generated.
\end{theorem}

\subsection{Explicit Computation of Mordell-Weil Groups}%
\label{sub:explicit_computation_of_mordell_weil_groups}
If we suppose \ref{thm:mordell} holds, then we can use the
structure theorem for finitely generated groups \cite[Theorem 8.1]{sergeLangAlgebra}
the conclude that an elliptic curve $E$ over $\mathbb{Q}$ has
\begin{align*}
  E(\mathbb{Q}) \simeq \mathbb{Z}^{r} \times (\mathbb{Z}/d_1\mathbb{Z}) \times \cdots \times (\mathbb{Z}/d_n\mathbb{Z})
\end{align*}
for some integers $r, d_i$. We can moreover conclude this is the same as
\begin{align*}
  E(\mathbb{Q}) \simeq \mathbb{Z}^{r} \times E(\mathbb{Q})_{\tors}.
\end{align*}
We have gone into some examples of computing subgroups of $E(\mathbb{Q})_{\tors}$
in section \ref{sec:points_of_finite_order}, now we shall give
an example of how to
compute $r$ for a given curve.
\begin{example} \label{ex:explicitComputation}
  Let $E: y^2 = x^3 + (x - 1)^2$ be an elliptic curve over $\mathbb{Q}$.
  Recall from example \ref{ex:multiplicationIsogeny} that
  we can find maps $\Phi, \Psi$ such that for some elliptic curve
  $\bar{E}$ the diagram
  \begin{equation*}
  \begin{tikzcd}
    E(\mathbb{Q}) \ar[rr, bend left, "\begin{bmatrix}3\end{bmatrix}"] \ar[r, "\Phi"] & \bar{E}(\mathbb{Q}) \ar[r, "\Psi"] & E(\mathbb{Q})
  \end{tikzcd}
  \end{equation*}
  commutes.
  As in \cite[Section 3.1]{moniqueThesis} we can moreover define a map $\alpha: E(\mathbb{Q}) \to \mathbb{Q}^\times / \mathbb{Q}^{\times 3}$
  by mapping $P = (x, y)$ as
  \begin{align*}
    \alpha(P) =
    \begin{dcases}
      \mathbb{Q}^{\times 3} \:&\text{if}\, P = \mathbb{Q}, \\
      (y + (x - 1)) \mathbb{Q}^{\times 3}\:&\text{else}.
    \end{dcases}
  \end{align*}
  later on we will verify that this is
  a homomorphism and $\ker \alpha = \Psi \bar{E}(\mathbb{Q})$, for now we refer to \cite[Theorem 5]{moniqueThesis}.

  Knowing this, we find via the first isomorphism theorem
  \begin{align*}
    E(\mathbb{Q}) / \Psi \bar{E}(\mathbb{Q}) \simeq \alpha(E(\mathbb{Q}))
  \end{align*}
  and that $(x,y) \mapsto (x + y - 1) \mod \mathbb{Q}^{\times 3}$.

  Now we take an arbitrary rational point $p/q$ on $E$,
  then we should have
  \begin{align*}
    \frac{p^2}{q^2} = \frac{p^3}{q^3} + \frac{(p - q)^2}{q^2} &\iff p^3 + q^3 = 2pq^2 \\
                                                              &\iff \frac{p^3 + pp^2 q - 2pq^2 + q^3}{q^3} = \frac{p^2}{q^2}
  \end{align*}
  These being reduces fractions means that necessarily $q^3$ is a square,
  thus $y = m/e^2$ for some $m, e \in \mathbb{Z}$.
  Giving us
  \begin{align*}
    m^2 &= \frac{p^6}{q^2} + \frac{2p^5}{q} - 3 p^4 - 2p^3q + 6p^2 q^2 - 4 pq^3 + q^{4}
  \end{align*}
  So under $\alpha$ we have
  \begin{align*}
    (p/e^2, m/e^3) &\mapsto p/e^2 + m/e^3 - 1 \mod \mathbb{Q}^{\times 3} \\
                   &\equiv (pe + m - e^3)/e^3 \mod \mathbb{Q}^{\times 3} \\
                   &\equiv m + pe - e^3 \mod \mathbb{Q}^{\times 3}
  \end{align*}
  so we can factor
  \begin{align*}
    (-x^2 , y^2) &\mapsto -x^2 + y^2 - 1 \mod \mathbb{Q}^{\times 3} \\
                 &\equiv (y + x - 1)(y - (x - 1)) \mod \mathbb{Q}^{\times 3} \\
                 &\equiv (m + te - e^3)(m - te + e^3) \mod \mathbb{Q}^{\times 3}
  \end{align*}


\end{example}
